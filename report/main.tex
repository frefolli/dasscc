\documentclass[a4paper,11pt,oneside, table]{article}
\usepackage[margin=1in]{geometry}
\usepackage{setspace}
\usepackage{imakeidx}
\usepackage{float}
\usepackage{graphicx}
\usepackage{pdfpages}
\usepackage{csquotes}
\usepackage{caption}
\captionsetup[table]{labelfont=it}
\usepackage{pifont}% http://ctan.org/pkg/pifont

\newcommand{\cmark}{\ding{51}}%
\newcommand{\xmark}{\ding{55}}%

\usepackage{listings}
\usepackage{listings-cpp}
\usepackage{algorithm}
\usepackage{algpseudocode}

\newtheorem{nota}{Nota}

\usepackage[italian]{babel}
\usepackage[
  backend=bibtex,
  style=numeric,
  sorting=ydnt
  ]{biblatex}
\addbibresource{refs.bib}
\makeindex

\newcommand{\putimage}[4] {
	\begin{figure}[H]
	    \centering
	    \includegraphics[width={#4}\linewidth]{#1}
	    \caption{#2}\label{#3}
	\end{figure}
}

\newcommand{\putsubimage}[5] {
  \begin{minipage}{{#4}\linewidth}
	    \centering
      \includegraphics[width={#5}\linewidth]{#1}
	    \caption{#2}\label{#3}
	\end{minipage}
}

\newcommand{\putimagecouple}[2] {
  \begin{figure}[!htb]
      \centering
      #1
      \hspace{0.5cm}
      #2
  \end{figure}
}

\newcommand{\putimagequadruple}[4] {
  \begin{figure}[!htb]
      \centering
      #1
      \hspace{0.5cm}
      #2
      \linebreak
      #3
      \hspace{0.5cm}
      #4
  \end{figure}
}

\begin{document}
    \begin{titlepage}
        \noindent
        \begin{minipage}[t]{0.19\textwidth}
            \vspace{-4mm}{\includegraphics[scale=1.15]{logo_unimib.pdf}}
        \end{minipage}
        \begin{minipage}[t]{0.81\textwidth}
        {
                \setstretch{1.42}
                {\textsc{Università degli Studi di Milano - Bicocca}} \\
                \textbf{Scuola di Scienze} \\
                \textbf{Dipartimento di Informatica, Sistemistica e Comunicazione} \\
                \textbf{Corso di laurea magistrale in Informatica} \\
                \par
        }
        \end{minipage}
    	\vspace{40mm}
    	\begin{center}
            {\LARGE{
                    \setstretch{1.2}
                    \textbf{Relazione di Metodi del Calcolo Scientifico - Progetto 1 Bis}
                    \par
            }}
        \end{center}
        
        \vspace{50mm}
        
        \vspace{15mm}

        \begin{flushright}
            {\large \textbf{Relazione di:}} \\
            \large{Refolli Francesco}
            \large{865955}
        \end{flushright}
        
        \vspace{40mm}
        \begin{center}
            {\large{\bf Anno Accademico 2023-2024}}
        \end{center}
        \restoregeometry
    \end{titlepage}

    \printindex
    \tableofcontents
    \renewcommand{\baselinestretch}{1.5}

\section{Introduzione}

Lo scopo del progetto \`e realizzare una libreria per la risoluzione di matrici sparse simmetriche e definite positive utilizzando i seguenti metodi iterativi:

\begin{itemize}
    \item metodo di Jacobi
    \item metodo di Gauss-Seidel
    \item metodo del Gradiente
    \item metodo del Gradiente coniugato
\end{itemize}

I metodi implementati partono dal vettore iniziale nullo e si arrestano qualora la k-esima iterata $x^{(k)}$ soddisfi $\frac{\parallel Ax^{(k)}-b \parallel} {\parallel b \parallel} < tol$ , con \textit{tol} tolleranza assegnata dal utente. Oltre a questo controllo sulla soluzione $x^{(k)}$, i metodi iterativi tengono in considerazione un numero massimo di iterazioni al limite del quale si arrestano e segnalano di non essere giunti a convergenza. Il numero massimo di iterazioni scelto \`e $maxIter = 30000$.

\paragraph{Tecnologie utilizzate}

Per l'implementazione si \`e scelto di sviluppare la libreria in C++ con Eigen, una nota libreria Free and Open Source per la gestione di  matrici dense e sparse. Per verificare la qualit\`a delle implementazioni dei metodi sono stati costruiti dei test basati sui requisiti di convergenza dei metodi stesso (matrici a dominanza diagonale o simmetriche e definite positive).

\section{Architettura della libreria}

In generale tutti i metodi iterativi oggetto di questo progetto hanno la medesima struttura, per questa ragione si \`e deciso di strutturare la libreria in modo da avere un oggetto che si occupa dell'algoritmo generale e vari oggetti che implementano fasi specifiche dei singoli metodi iterativi. Siccome Gauss-Seidel richiede anche un solutore di matrici triangolari si \`e implementato anche quello da zero con Eigen.

\putimage{images/diagram.png}{Diagramma di Strutture e Tratti d'Interesse}{png:diagram_of_structures}{0.99}

\`E stato definito un \textbf{tratto} per un entit\`a \textbf{Engine} che pu\`o essere utilizzato dal risolutore \textbf{IterativeSolver} per espletare una risoluzione iterativa generica. Gli \textbf{Engine} manipolano uno \textbf{State} interno al risolutore per implementare le loro specificit\`a. I risolutori ritornano una monade \textbf{Result} che permette di indicare se la computazione \`e andata a buon fine (i.e. se il limite di iterazioni non \`e stato superato). Questo ha un duplice scopo: tracciare l'informazione banale della convergenza e permettere all'engine che esercita le funzioni del metodo di Gauss Seidel di imporre un'asserzione sul successo della risoluzione del sistema triangolare associato.

\begin{algorithm}
  \renewcommand\thealgorithm{}
  \caption{Algoritmo Generale Iterativo}
  \begin{algorithmic}
    \Procedure{IterativeSolver::run}{$A, b, tol, maxIter$}
      \State $state \gets State::new()$
      \State $state.A \gets A$
      \State $state.b \gets b$
      \State $norm\_of\_b \gets state.b.norm()$
      \State $state.x_k \gets ArrayOfZeros(state.A.cols())$
      \State $state.r_k \gets state.b - state.A \cdot state.x_k$
      \State engine.post\_initialize(state)
      \For{$k \gets 1$ to $maxIter$}
        \State $normalized\_residual \gets \frac {state.r_k.norm()} {norm\_of\_b}$
        \If{$normalized\_residual < tol$}
          \State \Return $Ok(state.r_k)$
        \EndIf
        \State engine.pre\_compute\_y(state)
        \State engine.compute\_y(state)
        \State $state.x_n = state.x_k + state.y$
        \State $state.r_n \gets state.b - state.A \cdot state.x_n$
        \State engine.post\_compute\_x(state)
        \State state.update()
      \EndFor
      \State \Return $Err()$
    \EndProcedure
  \end{algorithmic}
\end{algorithm}

\begin{nota}
L'istruzione \textit{state.update()} implementa il passaggio di iterazione, cosicch\`e si possa evitare di scrivere esplicitamente per ogni valore di iterata $x_k = x_n$ (dove chiaramente $n = k+1$). In Python ed altri linguaggi questo pu\`o essere fatto tramite la metaprogrammazione, in C++ si \`e fatto uso delle macro per semplificare questo passaggio all'interno di \textit{State}:

\begin{lstlisting}[language=C++]
#define AssignKToNAndCleanN(what) \
  what##_k = std::move(what##_n); \
  what##_n.resize(1, 1); \
  what##_n.data().squeeze()
\end{lstlisting}
\end{nota}

\pagebreak

\section{Le Matrici Allegate}

\putimagequadruple
  {\putsubimage{./images/spy-spa1.png}{Visualizzazione di spa1.mtx}{png:spy-spa1}{0.4}{0.99}}
  {\putsubimage{./images/spy-spa2.png}{Visualizzazione di spa2.mtx}{png:spy-spa2}{0.4}{0.99}}
  {\putsubimage{./images/spy-vem1.png}{Visualizzazione di vem1.mtx}{png:spy-vem1}{0.4}{0.99}}
  {\putsubimage{./images/spy-vem2.png}{Visualizzazione di vem2.mtx}{png:spy-vem2}{0.4}{0.99}}

Come si pu\`o notare dal plot delle entrate non nulle, le prime due matrici (figure \ref{png:spy-spa1} e \ref{png:spy-spa2}) non presentano caratteristiche particolari, se non il fatto di essere simmetriche e definite positive.
Al contrario le altre due (figure \ref{png:spy-vem1} e \ref{png:spy-vem2}) sono matrici simmetriche e definite positive a bande lungo la diagonale principale.

\putimage{./images/dominance.png}{Rapporti di Dominanza Diagonale (ordinati) delle matrici}{png:dominance}{0.9}

Computazionalmente \`e stato rilevato che le matrici $vem1$ e $vem2$ sono \textbf{a dominanza diagonale in senso debole} per colonne (e per righe essendo simmetriche), mentre le matrici $spa1$ e $spa2$ non lo sono (hanno almeno una riga/colonna che non soddisfa il criterio di dominanza diagonale debole).
In figura \ref{png:dominance} si riportano i rapporti di dominanza diagonale (per colonne) ordinati per magnitudine (ottenuti come $\frac {| M[i, i] |} {\Sigma _ {i \ne j} | M[i, j] |}$).

\section{Convergenza}

Sono state effettuate delle esecuzioni al variare della tolleranza, del metodo di risoluzione e della matrice di base $A$ (il problema \`e generato prendendo $x_e$ = $<1,.., 1>$ e quindi $b = Ax_e$) scelta tra le quattro matrici allegate al testo della consegna.
Si riportano gli andamenti del numero di iterazioni impiegati per completare la risoluzione con i vari metodi (in scala semilogaritmica sulla $Y$, per facilitarne la consultazione siccome i grafici lineari risultano di difficile lettura).
Si segnala fin da subito che tutte le matrici convergono per i vari metodi, risultato importante visto che la convergenza di Gauss Seidel e Jacobi sulle matrici simmetriche e definite positive non \`e certa.

\subsection{Metodi basati sul Gradiente}

I due metodi basati sul gradiente hanno un andamento simile: lavorano bene sulle due matrici a bande (vem1/vem2), mentre hanno qualche leggera difficolt\`a a convergere con tolleranze p\`u basse con le due matrici "dense" randomiche (spa1/spa2).
Un dato significativo \`e il numero di iterazioni necessarie, per il metodo del \textbf{Gradiente Coniugato} \`e di molto inferiore rispetto a quello del \textbf{Gradiente}. A prima vista pu\`o sembrare che il metodo del Gradiente Coniugato sia pi\`u lento per il fatto che esegue pi\`u operazioni per ogni iterazione rispetto al Gradiente, ma siccome il numero di iterazioni \`e complessivamente di scala pi\`u bassa, abbiamo una priva evidenza che suggerisce che sia pi\`u mediamente veloce.

\putimagecouple
  {\putsubimage{./images/it-re-cge.png}{Andamenti (logaritmici) del Numero di Iterazioni con il metodo del Gradiente Coniugato}{png:it-re-cge}{0.4}{0.99}}
  {\putsubimage{./images/it-re-gre.png}{Andamenti (logaritmici) del Numero di Iterazioni con il metodo del Gradiente}{png:it-re-gre}{0.4}{0.99}}

\putimagecouple
  {\putsubimage{./images/te-re-cge.png}{Andamenti (logaritmici) del Tempo di Esecuzione con il metodo del Gradiente Coniugato}{png:te-re-cge}{0.4}{0.99}}
  {\putsubimage{./images/te-re-gre.png}{Andamenti (logaritmici) del Tempo di Esecuzione con il metodo del Gradiente}{png:te-re-gre}{0.4}{0.99}}

\subsection{Metodi Stazionari}

Anche qui gli andamenti dei due metodi si assomigliano asintoticamente: le due matrici a dominanza diagonale convergono molto pi\`u lentamente delle due matrici randomiche. In particolare il metodo di \textbf{Jacobi} impiega mediamente il doppio delle iterazioni rispetto al metodo di \textbf{Gauss Seidel}, tuttavia se osserviamo i dati rispetto al tempo di esecuzione impiegato si scopre come questa differenza \`e bilanciata dalla semplicit\`a delle operazioni del metodo di Jacobi.

\putimagecouple
  {\putsubimage{./images/it-re-gse.png}{Andamenti (logaritmici) del Numero di Iterazioni con il metodo di Gauss Seidel}{png:it-re-gse}{0.4}{0.99}}
  {\putsubimage{./images/it-re-jae.png}{Andamenti (logaritmici) del Numero di Iterazioni con il metodo di Jacobi}{png:it-re-jae}{0.4}{0.99}}

\putimagecouple
  {\putsubimage{./images/te-re-gse.png}{Andamenti (logaritmici) del Tempo di Esecuzione con il metodo di Gauss Seidel}{png:te-re-gse}{0.4}{0.99}}
  {\putsubimage{./images/te-re-jae.png}{Andamenti (logaritmici) del Tempo di Esecuzione con il metodo di Jacobi}{png:te-re-jae}{0.4}{0.99}}

\section{Scalabilit\`a sulla Dimensione}

Le quattro matrici hanno dimensione diversa quindi gli andamenti soffrono anche della scala rispetto al numero di righe/colonne. Per verificare come scalano i metodi, per ognuno di essi sono state generate randomicamente delle matrici di base $A$ di dimensione variabile ($100 \rightarrow 10000$) con il requisito che soddisfino il criterio di convergenza e si sono annotati i tempi di esecuzione per valutarne l'andamento all'aumentare della dimensione della matrice di input.

\putimagecouple
  {\putsubimage{./images/001-benchmark-N-elapsed.png}{Andamenti (lineari) del Tempo di Esecuzione con matrici RDD}{png:001-benchmark-N-elapsed}{0.4}{0.99}}
  {\putsubimage{./images/002-benchmark-N-elapsed.png}{Andamenti (lineari) del Tempo di Esecuzione con matrici SPD}{png:002-benchmark-N-elapsed}{0.4}{0.99}}

In figura \ref{png:001-benchmark-N-elapsed} si riportano i risultati del benchmark su matrici a dominanza diagonale simmetriche e definite positive su tutti e quattro i metodi risoluzione.
I metodi stazionari convergono pi\`u in fretta al target ($10^-8$) dei metodi basati sul gradiente.

Invece in figura \ref{png:002-benchmark-N-elapsed} si riportano i risultati del benchmark su matrici simmetriche e definite positive sui metodi basati sul gradiente.
Come detto in precedenza il metodo del Gradient Coniugato nel caso di matrici randomiche SPD converge molto pi\`u velocemente del metodo del Gradiente a causa del basso numero di iterazioni necessarie che bilancia il peso di ogni iterazione.

\printbibliography[title={Bibliografia}]
\end{document}
