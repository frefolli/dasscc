\documentclass[a4paper,11pt,oneside, table]{article}
\usepackage[margin=1in]{geometry}
\usepackage{setspace}
\usepackage{imakeidx}
\usepackage{float}
\usepackage{graphicx}
\usepackage{pdfpages}
\usepackage{csquotes}
\usepackage{caption}
\captionsetup[table]{labelfont=it}
\usepackage{pifont}% http://ctan.org/pkg/pifont

\newcommand{\cmark}{\ding{51}}%
\newcommand{\xmark}{\ding{55}}%

\usepackage{listings}
\usepackage{listings-cpp}

\newtheorem{nota}{Nota}

\usepackage[italian]{babel}
\usepackage[
  backend=bibtex,
  style=numeric,
  sorting=ydnt
  ]{biblatex}
\addbibresource{refs.bib}
\makeindex

\newcommand{\putimage}[4] {
	\begin{figure}[H]
	    \centering
	    \includegraphics[width={#4}\linewidth]{#1}
	    \caption{#2}\label{#3}
	\end{figure}
}

\newcommand{\putsubimage}[5] {
  \begin{minipage}{{#4}\linewidth}
	    \centering
      \includegraphics[width={#5}\linewidth]{#1}
	    \caption{#2}\label{#3}
	\end{minipage}
}

\newcommand{\putimagecouple}[2] {
  \begin{figure}[!htb]
      \centering
      #1
      \hspace{0.5cm}
      #2
  \end{figure}
}

\newcommand{\putimagequadruple}[4] {
  \begin{figure}[!htb]
      \centering
      #1
      \hspace{0.5cm}
      #2
      \linebreak
      #3
      \hspace{0.5cm}
      #4
  \end{figure}
}

\begin{document}
    \begin{titlepage}
        \noindent
        \begin{minipage}[t]{0.19\textwidth}
            \vspace{-4mm}{\includegraphics[scale=1.15]{logo_unimib.pdf}}
        \end{minipage}
        \begin{minipage}[t]{0.81\textwidth}
        {
                \setstretch{1.42}
                {\textsc{Università degli Studi di Milano - Bicocca}} \\
                \textbf{Scuola di Scienze} \\
                \textbf{Dipartimento di Informatica, Sistemistica e Comunicazione} \\
                \textbf{Corso di laurea magistrale in Informatica} \\
                \par
        }
        \end{minipage}
    	\vspace{40mm}
    	\begin{center}
            {\LARGE{
                    \setstretch{1.2}
                    \textbf{Relazione di Metodi del Calcolo Scientifico - Progetto 1 Bis}
                    \par
            }}
        \end{center}
        
        \vspace{50mm}
        
        \vspace{15mm}

        \begin{flushright}
            {\large \textbf{Relazione di:}} \\
            \large{Refolli Francesco}
            \large{865955}
        \end{flushright}
        
        \vspace{40mm}
        \begin{center}
            {\large{\bf Anno Accademico 2023-2024}}
        \end{center}
        \restoregeometry
    \end{titlepage}

    \printindex
    \tableofcontents
    \renewcommand{\baselinestretch}{1.5}

\section{Introduzione}

Lo scopo del progetto \`e realizzare una libreria per la risoluzione di matrici sparse simmetriche e definite positive utilizzando i seguenti metodi iterativi:

\begin{itemize}
    \item metodo di Jacobi
    \item metodo di Gauss-Seidel
    \item metodo del Gradiente
    \item metodo del Gradiente coniugato
\end{itemize}

I metodi implementati partono dal vettore iniziale nullo e si arrestano qualora la k-esima iterata $x^{(k)}$ soddisfi $\frac{\parallel Ax^{(k)}-b \parallel} {\parallel b \parallel} < tol$ , con \textit{tol} tolleranza assegnata dal utente. Oltre a questo controllo sulla soluzione $x^{(k)}$, i metodi iterativi tengono in considerazione un numero massimo di iterazioni al limite del quale si arrestano e segnalano di non essere giunti a convergenza. Il numero massimo di iterazioni scelto \`e $maxIter = 30000$.

\paragraph{Tecnologie utilizzate}

Per l'implementazione si \`e scelto di sviluppare la libreria in C++ con Eigen, una nota libreria Free and Open Source per la gestione di  matrici dense e sparse. Per verificare la qualit\`a delle implementazioni dei metodi sono stati costruiti dei test basati sui requisiti di convergenza dei metodi stesso (matrici a dominanza diagonale o simmetriche e definite positive).

\section{Architettura della libreria}

In generale tutti i metodi iterativi oggetto di questo progetto hanno la medesima struttura, per questa ragione si \`e deciso di strutturare la libreria in modo da avere un oggetto che si occupa dell'algoritmo generale e vari oggetti che implementano fasi specifiche dei singoli metodi iterativi. Siccome Gauss-Seidel richiede anche un solutore di matrici triangolari si \`e implementato anche quello da zero con Eigen.

\putimage{images/diagram.png}{Diagramma di Strutture e Tratti d'Interesse}{png:diagram_of_structures}{0.9}

\`E stato definito un \textbf{tratto} per un entit\`a \textbf{Engine} che pu\`o essere utilizzato dal risolutore \textbf{IterativeSolver} per espletare una risoluzione iterativa generica. Gli \textbf{Engine} manipolano uno \textbf{State} interno al risolutore per implementare le loro specificit\`a. I risolutori ritornano una monade \textbf{Result} che permette di indicare se la computazione \`e andata a buon fine (i.e. se il limite di iterazioni non \`e stato superato). Questo ha un duplice scopo: tracciare l'informazione banale della convergenza e permettere all'engine che esercita le funzioni del metodo di Gauss Seidel di imporre un'asserzione sul successo della risoluzione del sistema triangolare associato.

\begin{lstlisting}[language=C++]
Result<Vector> IterativeSolver::run(Matrix &A, Vector &b,
                                    double_t tol, uint32_t maxIter) {
  State state;
  // Load Input
  state.A = A;
  state.b = b;

  // Precompute norm of b
  double_t norm_of_b = state.b.norm();

  // Initialize
  const uint32_t N = A.cols();
  ArrayOfZeros(state.x_k, N);
  state.r_k = state.b - state.A * state.x_k;
  engine.post_initialize(state);

  for (uint32_t k = 1; k <= maxIter; ++k) {
    // Check Tolerance
    double_t normalized_residual = state.r_k.norm() / norm_of_b;
    if (normalized_residual < tol) {
      return Result<Vector>::Ok(state.x_k);
    }

    // Compute Y
    engine.pre_compute_y(state);
    engine.compute_y(state);

    // Compute Xn
    state.x_n = state.x_k + state.y;
    state.r_n = state.b - (state.A * state.x_n);
    engine.post_compute_x(state);

    // End of Iteration
    state.update();
  } 
  return Result<Vector>::Err();
}
\end{lstlisting}

\begin{nota}
L'istruzione \textit{state.update()} implementa il passaggio di iterazione, cosicch\`e si possa evitare di scrivere esplicitamente per ogni valore di iterata $x_k = x_n$ (dove chiaramente $n = k+1$). In Python ed altri linguaggi questo pu\`o essere fatto tramite la metaprogrammazione, in C++ si \`e fatto uso delle macro per semplificare questo passaggio all'interno di \textit{State}:


\begin{lstlisting}[language=C++]
#define AssignKToNAndCleanN(what) \
  what##_k = std::move(what##_n); \
  what##_n.resize(1, 1); \
  what##_n.data().squeeze()
\end{lstlisting}
\end{nota}

\section{Benchmark}

\subsection{Le Matrici Allegate}

\putimagequadruple
  {\putsubimage{./images/spy-spa1.png}{Visualizzazione di spa1.mtx}{png:spy-spa1}{0.4}{0.8}}
  {\putsubimage{./images/spy-spa2.png}{Visualizzazione di spa2.mtx}{png:spy-spa2}{0.4}{0.8}}
  {\putsubimage{./images/spy-vem1.png}{Visualizzazione di vem1.mtx}{png:spy-vem1}{0.4}{0.8}}
  {\putsubimage{./images/spy-vem2.png}{Visualizzazione di vem2.mtx}{png:spy-vem2}{0.4}{0.8}}

Come si pu\`o notare dal plot delle entrate non nulle, le prime due matrici (figure \ref{png:spy-spa1} e \ref{png:spy-spa2}) non presentano caratteristiche particolari, se non il fatto di essere simmetriche e definite positive.
Al contrario le altre due (figure \ref{png:spy-vem1} e \ref{png:spy-vem2}) sono matrici simmetriche e definite positive a bande lungo la diagonale principale.

\putimage{./images/dominance.png}{Rapporti di Dominanza Diagonale (ordinati) delle matrici}{png:dominance}{0.99}

Computazionalmente \`e stato rilevato che le matrici $vem1$ e $vem2$ sono \textbf{a dominanza diagonale in senso debole} per colonne (e per righe essendo simmetriche), mentre le matrici $spa1$ e $spa2$ non lo sono (hanno almeno una riga/colonna che non soddisfa il criterio di dominanza diagonale debole).
In figura \ref{png:dominance} si riportano i rapporti di dominanza diagonale (per colonne) ordinati per magnitudine (ottenuti come $\frac {| M[i, i] |} {\Sigma _ {i \ne j} | M[i, j] |}$).

\subsection{Test di Convergenza}

Sono state effettuate delle esecuzioni al variare della tolleranza, del metodo di risoluzione e della matrice di base $A$ (il problema \`e generato prendendo $x_e$ = $<1,.., 1>$ e quindi $b = Ax_e$) scelta tra le quattro matrici allegate al testo della consegna.
Si riportano gli andamenti del numero di iterazioni impiegati per completare la risoluzione con i vari metodi (in scala semilogaritmica, $logY$, per facilitarne la consultazione siccome i grafici lineari risultano di difficile lettura).

\putimagequadruple
  {\putsubimage{./images/it-re-cge.png}{Andamenti (logaritmici) con il metodo del Gradiente Coniugato}{png:it-re-cge}{0.4}{0.8}}
  {\putsubimage{./images/it-re-gre.png}{Andamenti (logaritmici) con il metodo del Gradiente}{png:it-re-gre}{0.4}{0.8}}
  {\putsubimage{./images/it-re-gse.png}{Andamenti (logaritmici) con il metodo di Gauss Seidel}{png:it-re-gse}{0.4}{0.8}}
  {\putsubimage{./images/it-re-jae.png}{Andamenti (logaritmici) con il metodo di Jacobi}{png:it-re-jae}{0.4}{0.8}}

\putimagequadruple
  {\putsubimage{./images/it-re-spa1.png}{Andamenti (logaritmici) su spa1.mtx}{png:it-re-spa1}{0.4}{0.8}}
  {\putsubimage{./images/it-re-spa2.png}{Andamenti (logaritmici) su spa2.mtx}{png:it-re-spa2}{0.4}{0.8}}
  {\putsubimage{./images/it-re-vem1.png}{Andamenti (logaritmici) su vem1.mtx}{png:it-re-vem1}{0.4}{0.8}}
  {\putsubimage{./images/it-re-vem2.png}{Andamenti (logaritmici) su vem2.mtx}{png:it-re-vem2}{0.4}{0.8}}

\subsection{Test di Scalabilit\`a}

Per ogni metodo di risoluzione sono state generare randomicamente delle matrici di base $A$ di dimensione variabile ($100 \rightarrow 10000$) con il requisito che soddisfino il criterio di convergenza e si sono annotati i tempi di esecuzione per valutarne l'andamento all'aumentare della dimensione della matrice di input.

\subsection{Tenere d'Occhio il Condizionamento}

Sono stati eseguiti vari test senza variare la dimensione della matrice ed annotando il condizionamento per vedere come si comportano i metodi al variare di esso.

\printbibliography[title={Bibliografia}]
\end{document}
